\documentclass[12pt, a4paper, oneside]{article}\usepackage[]{graphicx}\usepackage[]{color}
%% maxwidth is the original width if it is less than linewidth
%% otherwise use linewidth (to make sure the graphics do not exceed the margin)
\makeatletter
\def\maxwidth{ %
  \ifdim\Gin@nat@width>\linewidth
    \linewidth
  \else
    \Gin@nat@width
  \fi
}
\makeatother

\definecolor{fgcolor}{rgb}{0.345, 0.345, 0.345}
\newcommand{\hlnum}[1]{\textcolor[rgb]{0.686,0.059,0.569}{#1}}%
\newcommand{\hlstr}[1]{\textcolor[rgb]{0.192,0.494,0.8}{#1}}%
\newcommand{\hlcom}[1]{\textcolor[rgb]{0.678,0.584,0.686}{\textit{#1}}}%
\newcommand{\hlopt}[1]{\textcolor[rgb]{0,0,0}{#1}}%
\newcommand{\hlstd}[1]{\textcolor[rgb]{0.345,0.345,0.345}{#1}}%
\newcommand{\hlkwa}[1]{\textcolor[rgb]{0.161,0.373,0.58}{\textbf{#1}}}%
\newcommand{\hlkwb}[1]{\textcolor[rgb]{0.69,0.353,0.396}{#1}}%
\newcommand{\hlkwc}[1]{\textcolor[rgb]{0.333,0.667,0.333}{#1}}%
\newcommand{\hlkwd}[1]{\textcolor[rgb]{0.737,0.353,0.396}{\textbf{#1}}}%

\usepackage{framed}
\makeatletter
\newenvironment{kframe}{%
 \def\at@end@of@kframe{}%
 \ifinner\ifhmode%
  \def\at@end@of@kframe{\end{minipage}}%
  \begin{minipage}{\columnwidth}%
 \fi\fi%
 \def\FrameCommand##1{\hskip\@totalleftmargin \hskip-\fboxsep
 \colorbox{shadecolor}{##1}\hskip-\fboxsep
     % There is no \\@totalrightmargin, so:
     \hskip-\linewidth \hskip-\@totalleftmargin \hskip\columnwidth}%
 \MakeFramed {\advance\hsize-\width
   \@totalleftmargin\z@ \linewidth\hsize
   \@setminipage}}%
 {\par\unskip\endMakeFramed%
 \at@end@of@kframe}
\makeatother

\definecolor{shadecolor}{rgb}{.97, .97, .97}
\definecolor{messagecolor}{rgb}{0, 0, 0}
\definecolor{warningcolor}{rgb}{1, 0, 1}
\definecolor{errorcolor}{rgb}{1, 0, 0}
\newenvironment{knitrout}{}{} % an empty environment to be redefined in TeX

\usepackage{alltt} % Paper size, default font size and one-sided paper
%\graphicspath{{./Figures/}} % Specifies the directory where pictures are stored
%\usepackage[dcucite]{harvard}
\usepackage{amsmath}
\usepackage{setspace}
\usepackage{pdflscape}
\usepackage{rotating}
\usepackage[flushleft]{threeparttable}
\usepackage{multirow}
\usepackage[comma, sort&compress]{natbib}% Use the natbib reference package - read up on this to edit the reference style; if you want text (e.g. Smith et al., 2012) for the in-text references (instead of numbers), remove 'numbers' 
\usepackage{graphicx}
%\bibliographystyle{plainnat}
\bibliographystyle{agsm}
\usepackage[colorlinks = true, citecolor = blue, linkcolor = blue]{hyperref}
%\hypersetup{urlcolor=blue, colorlinks=true} % Colors hyperlinks in blue - change to black if annoying
%\renewcommand[\harvardurl]{URL: \url}
\usepackage{listings}
\usepackage{color}
\lstset{backgroundcolor=\color{yellow}}
\IfFileExists{upquote.sty}{\usepackage{upquote}}{}
\begin{document}
\title{Data Science and R Programming}
\author{Rob Hayward}
\date{\today}
\maketitle
\subsection*{Introduction}

This is an introduction to data science and R programming. There are two Coursera courses.  The main clasess can be found \href{https://www.coursera.org/course/datascitoolbox}{Data science toolkit} and \href{https://class.coursera.org/rprog-004}{R Programming}.  

\subsection{Git}
Some rules for adding files
\begin{itemize}
\item git add . adds all new files
\item git add -u updates tracking for files that were changed or deleted
\item git add -A all new files and all changes.
\end{itemize}

If you try to mix different classes in a vector, R will coerce to the lowest common denominator. For example, if you mix numeric and character, you will get two characters; if you try to mix logical and numeric, you will get numeric, if you try to mix logical and character, you will get character. 

A list is a special object.  It can change class.  
\begin{knitrout}
\definecolor{shadecolor}{rgb}{0.969, 0.969, 0.969}\color{fgcolor}\begin{kframe}
\begin{alltt}
\hlstd{x} \hlkwb{<-} \hlkwd{list}\hlstd{(}\hlnum{1}\hlstd{,} \hlstr{"a"}\hlstd{,} \hlnum{TRUE}\hlstd{,} \hlnum{1} \hlopt{+} \hlnum{4}\hlstd{)}
\hlstd{x}
\end{alltt}
\begin{verbatim}
## [[1]]
## [1] 1
## 
## [[2]]
## [1] "a"
## 
## [[3]]
## [1] TRUE
## 
## [[4]]
## [1] 5
\end{verbatim}
\end{kframe}
\end{knitrout}

The elements are indexed by double brackets.  

\subsection{Factors}
Factors are integer vectors with a label. The level is determined by alphabetical order.  Table can be used on factors. "Unclass" will strip the class and take the factor down to an integer. The levels can be set using "level" argument. Elements of an object can usually be named with "name". Lists and matrices can have names.

\subsection{Subsetting}
There are a number of ways to sub-set
\begin{itemize}
\item \lstinline{[} always returns an object of the same class as the original.  Can select more than one element. 
\item \lstinline{[[} is used to extract elements of a list or data frame.  Can only extract a single element and the class may not be an element or a dataframe. 
\item \lstinline{$} used to extract named elements of a list or data frame.  Same attributes as above.
\end{itemize}
Subsetting a matrix will usually return a vector. However, a matrix can be returned by setting \lstinline{drop = FALSE}. This can be important if taking a column of a matrix.  You will get a vector rather than a matrix unless you set \lstinline{drop = FALSE}

\begin{knitrout}
\definecolor{shadecolor}{rgb}{0.969, 0.969, 0.969}\color{fgcolor}\begin{kframe}
\begin{alltt}
\hlstd{x} \hlkwb{<-} \hlkwd{list}\hlstd{(}\hlkwc{foo} \hlstd{=} \hlnum{1}\hlopt{:}\hlnum{4}\hlstd{,} \hlkwc{bar} \hlstd{=} \hlnum{0.6}\hlstd{)}
\hlstd{x[}\hlnum{1}\hlstd{]}
\end{alltt}
\begin{verbatim}
## $foo
## [1] 1 2 3 4
\end{verbatim}
\begin{alltt}
\hlstd{x[[}\hlnum{1}\hlstd{]]}
\end{alltt}
\begin{verbatim}
## [1] 1 2 3 4
\end{verbatim}
\begin{alltt}
\hlstd{x}\hlopt{$}\hlstd{bar}
\end{alltt}
\begin{verbatim}
## [1] 0.6
\end{verbatim}
\begin{alltt}
\hlstd{x[[}\hlstr{"bar"}\hlstd{]]}
\end{alltt}
\begin{verbatim}
## [1] 0.6
\end{verbatim}
\end{kframe}
\end{knitrout}

The \lstinline{[[]]} can be used with computed indices. 
\begin{knitrout}
\definecolor{shadecolor}{rgb}{0.969, 0.969, 0.969}\color{fgcolor}\begin{kframe}
\begin{alltt}
\hlstd{x} \hlkwb{<-} \hlkwd{list}\hlstd{(}\hlkwc{foo} \hlstd{=} \hlnum{1}\hlopt{:}\hlnum{4}\hlstd{,} \hlkwc{bar} \hlstd{=} \hlnum{0.6}\hlstd{,} \hlkwc{baz} \hlstd{=} \hlstr{"hello"}\hlstd{)}
\hlstd{name} \hlkwb{<-} \hlstr{"foo"}
\hlstd{x[[name]]}
\end{alltt}
\begin{verbatim}
## [1] 1 2 3 4
\end{verbatim}
\begin{alltt}
\hlstd{x}\hlopt{$}\hlstd{name}
\end{alltt}
\begin{verbatim}
## NULL
\end{verbatim}
\begin{alltt}
\hlstd{x}\hlopt{$}\hlstd{foo}
\end{alltt}
\begin{verbatim}
## [1] 1 2 3 4
\end{verbatim}
\end{kframe}
\end{knitrout}

\begin{knitrout}
\definecolor{shadecolor}{rgb}{0.969, 0.969, 0.969}\color{fgcolor}\begin{kframe}
\begin{alltt}
\hlstd{x} \hlkwb{<-} \hlkwd{list}\hlstd{(}\hlkwc{a} \hlstd{=} \hlkwd{list}\hlstd{(}\hlnum{10}\hlstd{,} \hlnum{12}\hlstd{,} \hlnum{14}\hlstd{),} \hlkwc{b} \hlstd{=} \hlkwd{c}\hlstd{(}\hlnum{3.14}\hlstd{,} \hlnum{2.18}\hlstd{))}
\hlstd{x[[}\hlkwd{c}\hlstd{(}\hlnum{1}\hlstd{,} \hlnum{2}\hlstd{)]]}
\end{alltt}
\begin{verbatim}
## [1] 12
\end{verbatim}
\begin{alltt}
\hlstd{x[[}\hlnum{1}\hlstd{]][[}\hlnum{3}\hlstd{]]}
\end{alltt}
\begin{verbatim}
## [1] 14
\end{verbatim}
\begin{alltt}
\hlstd{x[[}\hlkwd{c}\hlstd{(}\hlnum{2}\hlstd{,} \hlnum{1}\hlstd{)]]}
\end{alltt}
\begin{verbatim}
## [1] 3.14
\end{verbatim}
\end{kframe}
\end{knitrout}


\subsection{reading data}
One way to speed up reading of large files is to read a small amount and to then create a vector with the column classes to input into the \lstinline{colClasses} section. 

\begin{lstlisting}
initial <- read.table("datatable.txt", nrows = 100)
classes <- sapply(initial, class)
tabAll <- readtable("datatable.txt", colClasses = classes)
\end{lstlisting}






\end{document}
