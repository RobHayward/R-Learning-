\documentclass[12pt, a4paper, oneside]{article}\usepackage[]{graphicx}\usepackage[]{color}
%% maxwidth is the original width if it is less than linewidth
%% otherwise use linewidth (to make sure the graphics do not exceed the margin)
\makeatletter
\def\maxwidth{ %
  \ifdim\Gin@nat@width>\linewidth
    \linewidth
  \else
    \Gin@nat@width
  \fi
}
\makeatother

\definecolor{fgcolor}{rgb}{0.345, 0.345, 0.345}
\newcommand{\hlnum}[1]{\textcolor[rgb]{0.686,0.059,0.569}{#1}}%
\newcommand{\hlstr}[1]{\textcolor[rgb]{0.192,0.494,0.8}{#1}}%
\newcommand{\hlcom}[1]{\textcolor[rgb]{0.678,0.584,0.686}{\textit{#1}}}%
\newcommand{\hlopt}[1]{\textcolor[rgb]{0,0,0}{#1}}%
\newcommand{\hlstd}[1]{\textcolor[rgb]{0.345,0.345,0.345}{#1}}%
\newcommand{\hlkwa}[1]{\textcolor[rgb]{0.161,0.373,0.58}{\textbf{#1}}}%
\newcommand{\hlkwb}[1]{\textcolor[rgb]{0.69,0.353,0.396}{#1}}%
\newcommand{\hlkwc}[1]{\textcolor[rgb]{0.333,0.667,0.333}{#1}}%
\newcommand{\hlkwd}[1]{\textcolor[rgb]{0.737,0.353,0.396}{\textbf{#1}}}%

\usepackage{framed}
\makeatletter
\newenvironment{kframe}{%
 \def\at@end@of@kframe{}%
 \ifinner\ifhmode%
  \def\at@end@of@kframe{\end{minipage}}%
  \begin{minipage}{\columnwidth}%
 \fi\fi%
 \def\FrameCommand##1{\hskip\@totalleftmargin \hskip-\fboxsep
 \colorbox{shadecolor}{##1}\hskip-\fboxsep
     % There is no \\@totalrightmargin, so:
     \hskip-\linewidth \hskip-\@totalleftmargin \hskip\columnwidth}%
 \MakeFramed {\advance\hsize-\width
   \@totalleftmargin\z@ \linewidth\hsize
   \@setminipage}}%
 {\par\unskip\endMakeFramed%
 \at@end@of@kframe}
\makeatother

\definecolor{shadecolor}{rgb}{.97, .97, .97}
\definecolor{messagecolor}{rgb}{0, 0, 0}
\definecolor{warningcolor}{rgb}{1, 0, 1}
\definecolor{errorcolor}{rgb}{1, 0, 0}
\newenvironment{knitrout}{}{} % an empty environment to be redefined in TeX

\usepackage{alltt} % Paper size, default font size and one-sided paper
%\graphicspath{{./Figures/}} % Specifies the directory where pictures are stored
%\usepackage[dcucite]{harvard}
\usepackage{rotating}
\usepackage{setspace}
\usepackage{pdflscape}
\usepackage[flushleft]{threeparttable}
\usepackage{multirow}
\usepackage[comma, sort&compress]{natbib}% Use the natbib reference package - read up on this to edit the reference style; if you want text (e.g. Smith et al., 2012) for the in-text references (instead of numbers), remove 'numbers' 
\usepackage{graphicx}
%\bibliographystyle{plainnat}
\bibliographystyle{agsm}
\usepackage[colorlinks = true, citecolor = blue, linkcolor = blue]{hyperref}
%\hypersetup{urlcolor=blue, colorlinks=true} % Colors hyperlinks in blue - change to black if annoying
%\renewcommand[\harvardurl]{URL: \url}
\IfFileExists{upquote.sty}{\usepackage{upquote}}{}
\begin{document}
\title{R learning notes}
\author{Rob Hayward}
\date{\today}
\maketitle
\section{Using dates and time}
This comes from \href{http://www.noamross.net/blog/2014/2/10/using-times-and-dates-in-r---presentation-code.html}{Bonnie Dixon}. This is an overview of using dates and times. 
\begin{knitrout}
\definecolor{shadecolor}{rgb}{0.969, 0.969, 0.969}\color{fgcolor}\begin{kframe}
\begin{alltt}
dt1 <- \hlkwd{as.Date}(\hlstr{"2014-02-15"})
dt1
\end{alltt}
\begin{verbatim}
## [1] "2014-02-15"
\end{verbatim}
\begin{alltt}
dt2 <- \hlkwd{as.Date}(\hlstr{"04/20/2011"}, format = \hlstr{"%m/%d/%Y"})
dt2
\end{alltt}
\begin{verbatim}
## [1] "2011-04-20"
\end{verbatim}
\begin{alltt}
dt1 - dt2
\end{alltt}
\begin{verbatim}
## Time difference of 1032 days
\end{verbatim}
\begin{alltt}
dt2 + 10
\end{alltt}
\begin{verbatim}
## [1] "2011-04-30"
\end{verbatim}
\end{kframe}
\end{knitrout}

Create a vector of dates and find the difference between them. 
\begin{knitrout}
\definecolor{shadecolor}{rgb}{0.969, 0.969, 0.969}\color{fgcolor}\begin{kframe}
\begin{alltt}
three.dates <- \hlkwd{as.Date}(\hlkwd{c}(\hlstr{"2010-07-22"}, \hlstr{"2011-04-20"}, \hlstr{"2012-06-10"}))
three.dates
\end{alltt}
\begin{verbatim}
## [1] "2010-07-22" "2011-04-20" "2012-06-10"
\end{verbatim}
\begin{alltt}
\hlkwd{diff}(three.dates)
\end{alltt}
\begin{verbatim}
## Time differences in days
## [1] 272 417
\end{verbatim}
\end{kframe}
\end{knitrout}

Create a sequence of days
\begin{knitrout}
\definecolor{shadecolor}{rgb}{0.969, 0.969, 0.969}\color{fgcolor}\begin{kframe}
\begin{alltt}
six.weeks <- \hlkwd{seq}(dt1, length = 6, by = \hlstr{"week"})
six.weeks
\end{alltt}
\begin{verbatim}
## [1] "2014-02-15" "2014-02-22" "2014-03-01" "2014-03-08" "2014-03-15"
## [6] "2014-03-22"
\end{verbatim}
\begin{alltt}
six.weeks <- \hlkwd{seq}(dt1, length = 6, by = 14)
six.weeks
\end{alltt}
\begin{verbatim}
## [1] "2014-02-15" "2014-03-01" "2014-03-15" "2014-03-29" "2014-04-12"
## [6] "2014-04-26"
\end{verbatim}
\begin{alltt}
six.weeks <- \hlkwd{seq}(dt1, length = 6, by = \hlstr{"2 weeks"})
six.weeks
\end{alltt}
\begin{verbatim}
## [1] "2014-02-15" "2014-03-01" "2014-03-15" "2014-03-29" "2014-04-12"
## [6] "2014-04-26"
\end{verbatim}
\end{kframe}
\end{knitrout}

\begin{knitrout}
\definecolor{shadecolor}{rgb}{0.969, 0.969, 0.969}\color{fgcolor}\begin{kframe}
\begin{alltt}
\hlkwd{unclass}(dt1)
\end{alltt}
\begin{verbatim}
## [1] 16116
\end{verbatim}
\begin{alltt}
dt1
\end{alltt}
\begin{verbatim}
## [1] "2014-02-15"
\end{verbatim}
\end{kframe}
\end{knitrout}


\subsection{POSIXct}
This is for the use of times. 
\begin{knitrout}
\definecolor{shadecolor}{rgb}{0.969, 0.969, 0.969}\color{fgcolor}\begin{kframe}
\begin{alltt}
tm1 <- \hlkwd{as.POSIXct}(\hlstr{"2009-07-24 23:55:26"})
tm1
\end{alltt}
\begin{verbatim}
## [1] "2009-07-24 23:55:26 BST"
\end{verbatim}
\begin{alltt}
tm2 <- \hlkwd{as.POSIXct}(\hlstr{"25072013 08:32:07"}, format = \hlstr{"%d%m%Y %H:%M:%S"})
tm2
\end{alltt}
\begin{verbatim}
## [1] "2013-07-25 08:32:07 BST"
\end{verbatim}
\end{kframe}
\end{knitrout}

Specify the time zone
\begin{knitrout}
\definecolor{shadecolor}{rgb}{0.969, 0.969, 0.969}\color{fgcolor}\begin{kframe}
\begin{alltt}
tm3 <- \hlkwd{as.POSIXct}(\hlstr{"2010-12-01 11:42:03"}, tz = \hlstr{"GMT"})
tm3
\end{alltt}
\begin{verbatim}
## [1] "2010-12-01 11:42:03 GMT"
\end{verbatim}
\end{kframe}
\end{knitrout}

Some calculations with times. 
\begin{knitrout}
\definecolor{shadecolor}{rgb}{0.969, 0.969, 0.969}\color{fgcolor}\begin{kframe}
\begin{alltt}
tm3 > tm2
\end{alltt}
\begin{verbatim}
## [1] FALSE
\end{verbatim}
\begin{alltt}
tm1 + 30
\end{alltt}
\begin{verbatim}
## [1] "2009-07-24 23:55:56 BST"
\end{verbatim}
\begin{alltt}
tm1 - 20
\end{alltt}
\begin{verbatim}
## [1] "2009-07-24 23:55:06 BST"
\end{verbatim}
\begin{alltt}
tm1 - tm2
\end{alltt}
\begin{verbatim}
## Time difference of -1461 days
\end{verbatim}
\begin{alltt}
\hlkwd{Sys.time}()
\end{alltt}
\begin{verbatim}
## [1] "2014-02-15 21:48:39 GMT"
\end{verbatim}
\begin{alltt}
\hlkwd{difftime}(tm1, \hlkwd{as.POSIXct}(\hlstr{"1970-01-01 00:00:00"}, tz = \hlstr{"UTC"}, units = \hlstr{"secs"}))
\end{alltt}
\begin{verbatim}
## Time difference of 14450 days
\end{verbatim}
\end{kframe}
\end{knitrout}

\subsection{POSIXlt}
The 'ct' stands for \emph{calendar time} while 'lt' stands for \emph{local time}.  
\begin{knitrout}
\definecolor{shadecolor}{rgb}{0.969, 0.969, 0.969}\color{fgcolor}\begin{kframe}
\begin{alltt}
tm1.lt <- \hlkwd{as.POSIXlt}(\hlstr{"2013-07-24 23:55:26"})
tm1.lt
\end{alltt}
\begin{verbatim}
## [1] "2013-07-24 23:55:26"
\end{verbatim}
\begin{alltt}
\hlkwd{unclass}(tm1.lt)
\end{alltt}
\begin{verbatim}
## $sec
## [1] 26
## 
## $min
## [1] 55
## 
## $hour
## [1] 23
## 
## $mday
## [1] 24
## 
## $mon
## [1] 6
## 
## $year
## [1] 113
## 
## $wday
## [1] 3
## 
## $yday
## [1] 204
## 
## $isdst
## [1] 1
\end{verbatim}
\begin{alltt}
\hlkwd{unlist}(tm1.lt)
\end{alltt}
\begin{verbatim}
##   sec   min  hour  mday   mon  year  wday  yday isdst 
##    26    55    23    24     6   113     3   204     1
\end{verbatim}
\end{kframe}
\end{knitrout}

The components of the time object can be extracted. 
\begin{knitrout}
\definecolor{shadecolor}{rgb}{0.969, 0.969, 0.969}\color{fgcolor}\begin{kframe}
\begin{alltt}
tm1.lt$sec
\end{alltt}
\begin{verbatim}
## [1] 26
\end{verbatim}
\begin{alltt}
tm1.lt$wday
\end{alltt}
\begin{verbatim}
## [1] 3
\end{verbatim}
\end{kframe}
\end{knitrout}

Truncate or round off the time. 
\begin{knitrout}
\definecolor{shadecolor}{rgb}{0.969, 0.969, 0.969}\color{fgcolor}\begin{kframe}
\begin{alltt}
\hlkwd{trunc}(tm1.lt, \hlstr{"days"})
\end{alltt}
\begin{verbatim}
## [1] "2013-07-24"
\end{verbatim}
\begin{alltt}
\hlkwd{trunc}(tm1.lt, \hlstr{"mins"})
\end{alltt}
\begin{verbatim}
## [1] "2013-07-24 23:55:00"
\end{verbatim}
\end{kframe}
\end{knitrout}

There is information on the lubridate. 
\end{document}
